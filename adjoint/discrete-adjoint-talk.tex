\documentclass{beamer}

%%%%%%%%%%%%%%%%%%%%%%%%%%%%%%%%%%%%%%%%%%%%%%%%%%%%%%%%%%%%%%%%
%%%                  Themes and such                         %%%
%%%%%%%%%%%%%%%%%%%%%%%%%%%%%%%%%%%%%%%%%%%%%%%%%%%%%%%%%%%%%%%%
\mode<presentation>
{
  %\usetheme{Copenhagen}  
  \usetheme{Warsaw}  
  %\usetheme{Malmoe}  
   % \setbeamertemplate{headline}{}
  %make my huge toc fit on one slide (and not look horrible)
  %\setbeamerfont{subsection in toc}{series=\bfseries}
  %\setbeamerfont{subsection in toc}{size=\tiny,series=\bfseries}
}

%%%%%%%%%%%%%%%%%%%%%%%%%%%%%%%%%%%%%%%%%%%%%%%%%%%%%%%%%%%%%%%%
%%%                       Packages                           %%%
%%%%%%%%%%%%%%%%%%%%%%%%%%%%%%%%%%%%%%%%%%%%%%%%%%%%%%%%%%%%%%%%
\usepackage{multimedia}
\usepackage{multirow}
\usepackage{subfigure}
\usepackage{amsmath}
\usepackage{parskip}
\setlength{\parskip}{\smallskipamount} 
\usepackage{color}
\usepackage{tikz}
%\usepackage{tcolorbox}

% Define commands
 \newcommand{\half}{\ensuremath{\frac{1}{2}}}

 \newcommand{\bea}{\begin{eqnarray}}
 \newcommand{\eea}{\end{eqnarray}}
 \newcommand{\beq}{\begin{equation}}
 \newcommand{\eeq}{\end{equation}}
 \newcommand{\bed}{\begin{displaymath}}
 \newcommand{\eed}{\end{displaymath}}

 \newcommand{\pd}[2]{\dfrac{\partial #1}{\partial #2}}
 \newcommand{\pf}[2]{\dfrac{d #1}{d #2}}
 \newcommand{\pdt}[2]{\dfrac{\partial^2 #1}{\partial #2^2}}
 \newcommand{\pft}[2]{\dfrac{d^2 #1}{d #2^2}}
 \newcommand{\pdtno}[2]{\dfrac{\partial^2 #1}{\partial #2}}
 \newcommand{\pdd}[3]{\dfrac{\partial^2 #1}{\partial #2 \partial #3}}
 \newcommand{\pff}[3]{\dfrac{d^2 #1}{d #2 d #3}}

 \graphicspath{{../figures/}}

\makeatletter
\newenvironment{noheadline}{
    \setbeamertemplate{headline}{}
    \addtobeamertemplate{frametitle}{\vspace*{-1.5\baselineskip}}{}
}{}
\makeatother

%%%%%%%%%%%%%%%%%%%%%%%%%%%%%%%%%%%%%%%%%%%%%%%%%%%%%%%%%%%%%%%%
%%%                     Title Info                           %%%
%%%%%%%%%%%%%%%%%%%%%%%%%%%%%%%%%%%%%%%%%%%%%%%%%%%%%%%%%%%%%%%%

\title[\hspace{-0.2cm} Time Dependent Discrete Adjoint]
{
Adjoint-Based Derivative Evaluation Methods for Flexible Multibody Systems
}

\author[K. Boopathy and G. J. Kennedy]
{
  \Large {Komahan Boopathy} \\ 
  and \\
  \Large {Graeme Kennedy} \\
}

\institute
{
  \large Structures and Multidisciplinary Optimization Laboratory \\
  ~\\
  School of Aerospace Engineering\\
  Georgia Institute of Technology\\
 Atlanta, GA
}

\date
{
\small \today
}

\begin{document}
\setbeamertemplate{background}{
  \begin{tikzpicture}
    \node[opacity=.01,inner sep=0pt]{
     \includegraphics [height=\paperheight]{gt.png}};
  \end{tikzpicture}
}
% Now we install the new template for the following frames:
%\usebackgroundtemplate{%
%  \includegraphics[width=\paperwidth,height=\paperheight]{}}

\begin{frame}
  \titlepage
\end{frame}

\setbeamertemplate{background}{
  \begin{tikzpicture}
    \node[opacity=.003,inner sep=0pt]{
     \includegraphics [height=\paperheight]{gt.png}};
  \end{tikzpicture}
}
%\begin{frame}
%  \frametitle{Outline}
%  \tableofcontents
%\end{frame}

\begin{frame}[fragile] 
  \frametitle{Motivation: Need for Derivatives}
  \begin{minipage}{\linewidth}
  \begin{minipage}{0.5\linewidth}\scriptsize{
      \begin{block}{Finite differences and complex-step}   
          \begin{itemize}
          \item Simple, robust but inefficient
          \item Doesn't require knowledge of governing equations
          \end{itemize}
      \end{block}
      \begin{block}{Adjoint and direct method}
      \begin{itemize}
      \item Efficient, robust but complicated
      \item Precise to the tolerance to which the governing ODEs/DAEs/PDEs
        are solved $(10^{-8}, 10^{-12}, 10^{-16})$
      \item Direct sensitivities are beyond scope
      \item Requires knowledge of governing equations (semi-analytic)
        \begin{enumerate}
        \item \scriptsize{Patience}
        \item \scriptsize{Calculus}
        \item \scriptsize{Linear algebra}
        \end{enumerate}
      \item Adjoint details, to follow...
      \end{itemize}
      \end{block}}
  \end{minipage}
  \begin{minipage}{0.6\linewidth}
    \begin{figure}
      \centering
      \includegraphics[trim={2pt 0 0 0}, clip,width=0.75\textwidth]{fd-csd.pdf}
    \end{figure}
    \end{minipage}
  \end{minipage}
\end{frame}

\begin{noheadline}
\begin{frame}[allowframebreaks] 
  \frametitle{Newmark--Beta--Gamma Method Adjoint}
  \scriptsize{

\begin{minipage}{1.0\linewidth}
\begin{block}{101. Notation:}
    \begin{tabbing}
      XXXXXX \= xxxx\kill
      $t$     \> time \\
      $k$     \> time index \\
      $x$     \> design variables \\
      $m$     \> number of design variables \\
      $u,v,w$ \> state variables \\
      $n$  \> number of state variables \\
      $R, S, T$ \> constraint governing equations  \\
      $\lambda,\psi,\phi$ \> adjoint variables  \\
      $F$     \> objective function \\
      $\cal{L}$ \> Lagrangian \\
    \end{tabbing}
\end{block}
\end{minipage}
\begin{minipage}{1.0\linewidth}
\begin{block}{1. Objective Function:}
    A time-averaged objective function is considered (e.g. mean lift,
    mean potential energy, mean thermal-stress). We can approximate
    the integral as a discrete sum:
    \begin{equation}\label{eqn:time-averaged-function}
      F = \frac{1}{T}\int_{0}^T f_k(u,v,w,x,t)~dt=\sum_{k=0}^N h f_k(u_k,v_k,w_k,x)
    \end{equation}
    Other possibilities for the choice of objective function exist
    too.
\end{block}
\end{minipage}
\framebreak

\begin{block}{2. Optimization Problem:}
  We would like to minimize the time-dependent objective function,
  $F$, such that the governing constraint equations $(R, S, T)$ are
  satisfied at all time steps.  Mathematically, a \underline{general}
  optimization problem can be posed as follows:
  \begin{equation}
    \begin{aligned}
      & \underset{x}{\text{minimize}}
      & & F = F(u, v, w, x,t) \\
      & \text{subject to} & &  R(u, x_r, t), S(v, x_s, t), T(w, x_t, t) = 0.\\
      %        & & &  S(v) = 0,\\
      %        & &  & T(w) = 0.\\
      %& \text{bounds}
      %& & \d^\text{L} \le \d \le \d^\text{U}.
    \end{aligned}
  \end{equation}
  where $x = [x_r, x_s, x_t]$ is a vector of design variables.
\end{block}

\begin{block}{3. Lagrangian:}
  It would be nice and handy to pack all separate equations (objects)
  into one equation (object) to operate with. We introduce
  \textcolor{red}{$\lambda_k$}, \textcolor{orange}{$\psi_k$} and
  \textcolor{blue}{$\phi_k$} as the adjoint variables associated with
  each of these function objects at the each time step and the
  Lagrangian function is written as:
  \begin{equation}\label{eqn:nbg-lagrangian}
    {\cal{L}} = \textcolor{brown}{\sum_{k=0}^N h f_k} + \textcolor{red}{ \sum_{k=0}^N h \lambda_k^T R_k} 
    +  \textcolor{orange}{ \sum_{k=0}^N\psi_k^T S_k} +  \textcolor{blue}{ \sum_{k=0}^N\phi_k^T T_k}.
  \end{equation}
  The Lagrangian is nothing but a linear ``\texttt{\underline{F}AXPY}'' 
  operator  (why not!) operating on vector objects -- just like \texttt{\underline{S}AXPY}, 
  \texttt{\underline{D}AXPY}, \texttt{\underline{Z}AXPY} (being Fortranic).
  Here $\lambda,~\phi,~\psi$ adjoint variables: they
  manifest themselves as simple-scaling-scalars in linear-algebra and
  as Lagrange multipliers in the context of optimization.
\end{block}


\framebreak

\begin{block}{4. Adjoint Variables:}
  This involves two main steps:
  \begin{enumerate}
  \item Forward mode for finding the state variables involving non-linear solves (2/3 of the
    computational time)
  \item Reverse mode for finding the adjoint variables involving
    linear-solves (1/3 of the computational time). The
    conditions $$\textcolor{red}{\pd{\cal{L}}{u_k}},
    \textcolor{orange}{\pd{\cal{L}}{v_k}},
    \textcolor{blue}{\pd{\cal{L}}{w_k}} = 0$$ can be used to generate
    a typically coupled system of equations to solve for the adjoint
    variables, $\textcolor{red}{\lambda}$, $\textcolor{orange}{\psi}$
    and $\textcolor{blue}{\phi}$. 
  \end{enumerate}

\end{block}

\begin{block}{Remark}
  We will get three vectors of equations for the three vectors of
  adjoint variable unknowns. Possibilities for decoupling the system
  exist -- case by case basis. 
\end{block}

\begin{block}{Remark}
  When decoupling is possible one or more of the
  constraint equations fold into other equations and therefore the
  corresponging adjoint variable need not be included in the
  formulation of the Lagrangian, but is typically done for the sake of
  convenience/generality of the implementation.
\end{block}

\begin{block}{Matrix Structure}
  \begin{minipage}{0.49\linewidth}
    \begin{figure}
      \centering
      \includegraphics[width=\textwidth]{general-forward-matrix.pdf}
      \label{Forward Mode}
    \end{figure}
  \end{minipage}
  \begin{minipage}{0.49\linewidth}
    \begin{figure}
      \centering
      \includegraphics[width=0.925\textwidth]{general-reverse-matrix.pdf}
      \label{Reverse Mode}
    \end{figure}
  \end{minipage}
\end{block}

\framebreak

  \begin{block}{5. Total Derivative:}
    The total derivative can be
    written as a \underline{linear combination} of contributions from the function
    of interest and the constraints:
    \begin{equation}\label{eqn:nbg-total-derivative}
      \pd{\cal{L}}{x} = \pd{F}{x} = 
      \textcolor{brown}{\sum_{k=0}^N h \pd{f_k}{x}^T} + 
      \textcolor{red}{\sum_{k=0}^N h \pd{R_k}{x}^T \lambda_k} + 
      \textcolor{orange}{\sum_{k=0}^N \pd{S_k}{x}^T \psi_k} + 
      \textcolor{blue}{\sum_{k=0}^N \pd{T_k}{x}^T \phi_k}.
    \end{equation}
  \end{block}
  %The projections of every other vector is used to 
  The objective function value $F$ and the gradient $\pd{F}{x}$ goes
  to the optimization code.

  \framebreak

  \begin{block}{Context of Structural Dynamics and Time Marching}
    \begin{minipage}{0.45\linewidth}
      \begin{tabbing}
        XXXXXX \= xxxx\kill
        $t$     \> time \\
        $k$     \> time index \\
        $x$     \> design variables \\
        $m$     \> number of design variables \\
        \colorbox{blue!20}{$u,v,w$} \> \colorbox{blue!20}{state variables} \\
        $n$  \> number of state variables \\
        \colorbox{green!20}{$R, S, T$} \> constraint governing equations  \\
        $\lambda ,~ \psi ,~\phi$ \> adjoint variables  \\
        $F$     \> objective function \\
        $\cal{L}$ \> Lagrangian \\
      \end{tabbing}
    \end{minipage}\hfill
    \begin{minipage}{0.45\linewidth}
      \begin{tabbing}
        XXXXXX \= xxxx\kill
        $t$     \> time \\
        $k$     \> time index \\
        $x$     \> design variables \\
        $m$     \> number of design variables \\
        \colorbox{blue!20}{$q,\dot{q},\ddot{q}$} \> \colorbox{blue!20}{position, velocity and acceleration states} \\
        $n$  \> number of state variables \\
        \colorbox{green!20}{$R, S, T$} \> constraint governing equations  \\
        $\lambda ,~ \psi ,~\phi$ \> adjoint variables  \\
        $F$     \> objective function \\
        $\cal{L}$ \> Lagrangian \\
      \end{tabbing}
    \end{minipage}
  \end{block}

  The residual of the governing equations from structural dynamics
  can be posed as: 
  \begin{equation}\label{eqn:nbg-residual}
    \textcolor{red}{R_k = R_k(\underline{\ddot{q}_k}, \dot{q}_k, q_k, x) = 0}.
  \end{equation}
  \begin{example} 
    $R = m\ddot{q} + c\dot{q} + k{q} = 0$, x = $[m,
      c, k]$ can be the design variables of choice, the objective be
    minimizing the time-averaged potential energy $F = \frac{1}{T}
    \int_{0}^T f_k~dt$, where $f_k = \frac{1}{2} kq^2$ is the
    instantaneous PE.
  \end{example}
}
\end{frame}

\begin{frame}\frametitle{Newmark--Beta--Gamma Adjoint}
\scriptsize{
  \begin{minipage}{0.6\linewidth}
    We use \textbf{Newmark--Beta--Gamma (NBG)} time-marching method to
    integrate and solve for the states over time.  The states
    approximation equations are:
    \begin{equation}\label{eqn:nbg-approx-qdot}
      \textcolor{orange}{S_k =  \dot{q}_{k-1}  + (1-\gamma) h \ddot{q}_{k-1} +  \gamma h \ddot{q}_{k}- \dot{q}_k=0, }
    \end{equation}
    \begin{equation}\label{eqn:nbg-approx-q}
      \textcolor{blue}{T_k = {q}_{k-1} + h \dot{q}_{k-1} +\frac{1-2\beta}{2}
        h^2\ddot{q}_{k-1} + \beta h^2 \ddot{q}_k-{q}_k=0}. 
    \end{equation}
    The underlined varibles are the primary variables in each equation.
    Eq~\eqref{eqn:nbg-residual} comes from the governing physics of
    the system. Eqs.~\eqref{eqn:nbg-approx-qdot} and
    ~\eqref{eqn:nbg-approx-q} come from the time-marching
    scheme. Therefore, the equations $S_k$ and $T_k$ are independent
    of the design variables $x$ -- they do not contribute to the total
    derivative. 
  \end{minipage}
  \begin{minipage}{0.39\linewidth}
    \begin{figure}
      \centering
      \includegraphics[width=\textwidth]{nbg-lagrangian.pdf}
      %\caption{Illustrations showing the working of NBG method.}
      \label{fig:nbg-illustration}
    \end{figure}
  \end{minipage}
  
  \textbf{Total Derivative:}
  \begin{equation}\label{eqn:nbg-total-derivative}
    \pd{\cal{L}}{x} = \pd{F}{x} = \textcolor{brown}{\sum_{k=0}^N h \pd{f_k}{x}^T} +\textcolor{red}{ \sum_{k=0}^N h
      \pd{R_k}{x}^T \lambda_k}.
  \end{equation}

  %    \begin{minipage}{1.0\linewidth}
  %      \begin{figure}
  %        \centering
  %        \includegraphics[width=0.925\textwidth]{nbg-reverse-matrix.pdf}
  %        \label{Reverse Mode}
  %      \end{figure}
  %    \end{minipage}

  %    \begin{itemize}

  %    \item Since the equations $S_k$ and $T_k$ are explicit, it becomes
  %      possible to solve \underline{sequentially} for the adjoint
  %      variables at each step. This property is beneficial in terms of
  %      reducing the size of the linear system. The matrix is upper
  %      triangular and we can solve for the adjoint variables sequentially
  %      via back-substitution -- we march backwards in time. The above
  %      treatment is based on the application of chain rule of
  %      differentiation. 

  %   \item Differentiate the terms that carry information within the
  %     current-step from the ones that carry them across time-steps.
  \textbf{Adjoint Equations:}
  The system of equations for the adjoint variables is given by
  $$\textcolor{red}{\pd{\cal{L}}{\ddot{q}_k}},
  \textcolor{orange}{\pd{\cal{L}}{\dot{q}_k}},
  \textcolor{blue}{\pd{\cal{L}}{q_k}} = 0.$$      

  %    \end{itemize}
  \begin{itemize}
  \item Solving for $\phi$ using $\pd{\cal{L}}{{q}_k} = 0$:
    \begin{equation}
      \begin{split}
        \phi_k = \phi_{k+1} + h \left\{ \pd{f_{k+1}}{{q}_{k+1}} \right\}^T + h \left[\pd{R_{k+1}}{{q}_{k+1}} \right]^T \lambda_{k+1}  
      \end{split}
    \end{equation}

  \item Solving for $\psi_k$ using $\pd{\cal{L}}{\dot{q}_k} = 0$:
    \begin{equation}
      \begin{split}
        \psi_k = \psi_{k+1} + h \phi_{k+1}  +h \left\{ \pd{f_{k+1}}{\dot{q}_{k+1}} +  h \pd{f_{k+1}}{{q}_{k+1}} \right\}^T + h\left[ \pd{R_{k+1}}{\dot{q}_{k+1}} +  h \pd{R_{k+1}}{{q}_{k+1}} \right]^T \lambda_{k+1} 
      \end{split}
    \end{equation}

  \item Solving for $\lambda_k$ using $\pd{\cal{L}}{\ddot{q}_k} = 0$:
    
    \begin{equation}
      \begin{split}
        \left[ \pd{R_k}{\ddot{q}_k} + \gamma h \pd{R_k}{\dot{q}_k} + \beta h^2 \pd{R_k}{{q}_k} \right]^T \lambda_k = &- \left\{ \pd{f_k}{\ddot{q}_k} + \gamma h \pd{f_k}{\dot{q}_k} + \beta h^2 \pd{f_k}{{q}_k} \right\}^T \\
        & -  \frac{1}{h}\left\{  \gamma h  \psi_k + \beta h^2   \phi_k \right\}^T\\
        & -  \left\{ (1-\gamma) h \pd{f_{k+1}}{\dot{q}_{k+1}} + \frac{1-2\beta}{2} h^2 \pd{f_{k+1}}{{q}_{k+1}} \right\}^T \\
        & -  \left[ (1-\gamma) h \pd{R_{k+1}}{\dot{q}_{k+1}} + \frac{1-2\beta}{2} h^2 \pd{R_{k+1}}{{q}_{k+1}} \right]^T\lambda_{k+1} \\
        & -  \frac{1}{h} \left\{ (1-\gamma) h \psi_{k+1} + \frac{1-2\beta}{2} h^2 \phi_{k+1} \right\}^T\\
      \end{split}
    \end{equation}
  \end{itemize}
  
  Figure~\ref{fig:nbg-illustration} shows the flow of information across
  states.
  \begin{figure}
    \centering
    \includegraphics[width=\textwidth]{nbg-chart.pdf}
    %\caption{Illustrations showing the working of NBG method.}
    \label{fig:nbg-illustration}
  \end{figure}

}
\end{frame}
\end{noheadline}
\end{document}

%         We explore further simplifications by substituting equations.
%
%         \begin{equation}
%           \begin{split}
%             \left[ \frac{1}{h^2} \pd{R_k}{\ddot{q}_k} + \gamma \frac{1}{h} \pd{R_k}{\dot{q}_k} + \beta \pd{R_k}{{q}_k} \right]^T \lambda_k = &- \left\{ \frac{1}{h^2}  \pd{f_k}{\ddot{q}_k} + \gamma \frac{1}{h} \pd{f_k}{\dot{q}_k} + \beta \pd{f_k}{{q}_k} \right\}^T \\
%             & -  \frac{1}{h}\left\{  \gamma \frac{1}{h}  \left( \psi_{k+1} + h \phi_{k+1}  + \left\{ h \pd{f_{k+1}}{\dot{q}_{k+1}} +  h^2 \pd{f_{k+1}}{{q}_{k+1}} \right\}^T + \left[ h \pd{R_{k+1}}{\dot{q}_{k+1}} +  h^2 \pd{R_{k+1}}{{q}_{k+1}} \right]^T \lambda_{k+1}  \right) \right\}^T \\
%             & -  \frac{1}{h}\left\{  \beta  \left( \phi_{k+1} + \left\{h \pd{f_{k+1}}{{q}_{k+1}} \right\}^T + \left[ h \pd{R_{k+1}}{{q}_{k+1}} \right]^T \lambda_{k+1}  \right) \right\}^T\\
%             & -  \left\{ (1-\gamma) \frac{1}{h} \pd{f_{k+1}}{\dot{q}_{k+1}} + \frac{1-2\beta}{2} \pd{f_{k+1}}{{q}_{k+1}} \right\}^T \\
%             & -  \left[ (1-\gamma) \frac{1}{h} \pd{R_{k+1}}{\dot{q}_{k+1}} + \frac{1-2\beta}{2} \pd{R_{k+1}}{{q}_{k+1}} \right]^T\lambda_{k+1} \\
%             & -  \frac{1}{h} \left\{ (1-\gamma) \frac{1}{h} \psi_{k+1} + \frac{1-2\beta}{2} \phi_{k+1} \right\}^T\\
%           \end{split}
%         \end{equation}
%
%         \framebreak


         \framebreak
        
         Grouping the terms together we get:

         \begin{equation}
           \begin{split}
             \left[ \frac{1}{h^2} \pd{R_k}{\ddot{q}_k} + \gamma \frac{1}{h} \pd{R_k}{\dot{q}_k} + \beta \pd{R_k}{{q}_k} \right]^T \lambda_k = &- \left\{ \frac{1}{h^2}  \pd{f_k}{\ddot{q}_k} + \gamma \frac{1}{h} \pd{f_k}{\dot{q}_k} + \beta \pd{f_k}{{q}_k} \right\}^T \\
             & -  \left\{  \gamma \frac{1}{h}  \pd{f_{k+1}}{\dot{q}_{k+1}} +  \gamma \pd{f_{k+1}}{{q}_{k+1}} \right\}^T \\ 
             & -  \left[  \gamma \frac{1}{h}  \pd{R_{k+1}}{\dot{q}_{k+1}} + \gamma \pd{R_{k+1}}{{q}_{k+1}} \right]^T \lambda_{k+1}  \\
             & -  \left\{ \beta \pd{f_{k+1}}{{q}_{k+1}} \right\}^T \\ 
             & - \left[  \beta\pd{R_{k+1}}{{q}_{k+1}} \right]^T \lambda_{k+1}  \\
             & -  \left\{ (1-\gamma) \frac{1}{h} \pd{f_{k+1}}{\dot{q}_{k+1}} + (\frac{1}{2}-\beta) \pd{f_{k+1}}{{q}_{k+1}} \right\}^T \\
             & -  \left[ (1-\gamma) \frac{1}{h} \pd{R_{k+1}}{\dot{q}_{k+1}} + (\frac{1}{2}-\beta) \pd{R_{k+1}}{{q}_{k+1}} \right]^T\lambda_{k+1} \\
             & -  \frac{1}{h} \left\{ \frac{1}{h} \psi_{k+1} + (\frac{1}{2} +\gamma) \phi_{k+1} \right\}^T\\
           \end{split}
         \end{equation}
        
         \begin{equation}
           \begin{split}
             \left[ \frac{1}{h^2} \pd{R_k}{\ddot{q}_k} + \gamma \frac{1}{h} \pd{R_k}{\dot{q}_k} + \beta \pd{R_k}{{q}_k} \right]^T \lambda_k = &- \left\{ \frac{1}{h^2}  \pd{f_k}{\ddot{q}_k} + \gamma \frac{1}{h} \pd{f_k}{\dot{q}_k} + \beta \pd{f_k}{{q}_k} \right\}^T \\
             & -  \left\{  \frac{1}{h} \pd{f_{k+1}}{\dot{q}_{k+1}} +  (\frac{1}{2} +\gamma) \pd{f_{k+1}}{{q}_{k+1}} \right\}^T \\
             & -  \left[  \frac{1}{h} \pd{R_{k+1}}{\dot{q}_{k+1}} +  (\frac{1}{2} +\gamma)  \pd{R_{k+1}}{{q}_{k+1}} \right]^T\lambda_{k+1} \\
             & -  \frac{1}{h} \left\{ \frac{1}{h} \psi_{k+1} + (\frac{1}{2} +\gamma) \phi_{k+1} \right\}^T\\
           \end{split}
         \end{equation}

         Once the primary adjoint variables $\lambda_k$ have been
         determined, the total derivative is readily obtained using
         Eq.\eqref{eqn:nbg-total-derivative}:
         $$\pd{\cal{L}}{x} = \pd{F}{x} = \sum_{k=0}^N h \pd{f_k}{x}^T
         + \sum_{k=0}^N h \pd{R_k}{x}^T \lambda_k.$$ Notice that
         $\pd{S_k}{x} = \pd{T_k}{x} = 0$.  
  


         We can equivalently scale the equation with $1/h^2$ and represent as follows:
         
         \begin{equation}
           \begin{split}
             \left[ \frac{1}{h^2} \pd{R_k}{\ddot{q}_k} + \gamma \frac{1}{h} \pd{R_k}{\dot{q}_k} + \beta \pd{R_k}{{q}_k} \right]^T \lambda_k = &- \left\{ \frac{1}{h^2}  \pd{f_k}{\ddot{q}_k} + \gamma \frac{1}{h} \pd{f_k}{\dot{q}_k} + \beta \pd{f_k}{{q}_k} \right\}^T \\
             & -  \frac{1}{h}\left\{  \gamma \frac{1}{h}  \psi_k + \beta   \phi_k \right\}^T\\
             & -  \left\{ (1-\gamma) \frac{1}{h} \pd{f_{k+1}}{\dot{q}_{k+1}} + \frac{1-2\beta}{2} \pd{f_{k+1}}{{q}_{k+1}} \right\}^T \\
                          & -  \left[ (1-\gamma) \frac{1}{h} \pd{R_{k+1}}{\dot{q}_{k+1}} + \frac{1-2\beta}{2} \pd{R_{k+1}}{{q}_{k+1}} \right]^T\lambda_{k+1} \\
             & -  \frac{1}{h} \left\{ (1-\gamma) \frac{1}{h} \psi_{k+1} + \frac{1-2\beta}{2} \phi_{k+1} \right\}^T\\
           \end{split}
         \end{equation}
