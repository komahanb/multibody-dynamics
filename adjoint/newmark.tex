\documentclass{beamer}

%%%%%%%%%%%%%%%%%%%%%%%%%%%%%%%%%%%%%%%%%%%%%%%%%%%%%%%%%%%%%%%%
%%%                  Themes and such                         %%%
%%%%%%%%%%%%%%%%%%%%%%%%%%%%%%%%%%%%%%%%%%%%%%%%%%%%%%%%%%%%%%%%
\mode<presentation>
{
  %\usetheme{Copenhagen}  
  %\usetheme{Warsaw}  
  \usetheme{Malmoe}  
%    \setbeamertemplate{headline}{}
  %make my huge toc fit on one slide (and not look horrible)
  %\setbeamerfont{subsection in toc}{series=\bfseries}
  %\setbeamerfont{subsection in toc}{size=\tiny,series=\bfseries}
}

%%%%%%%%%%%%%%%%%%%%%%%%%%%%%%%%%%%%%%%%%%%%%%%%%%%%%%%%%%%%%%%%
%%%                       Packages                           %%%
%%%%%%%%%%%%%%%%%%%%%%%%%%%%%%%%%%%%%%%%%%%%%%%%%%%%%%%%%%%%%%%%
\usepackage{multimedia}
\usepackage{multirow}
\usepackage{subfigure}
\usepackage{amsmath}

% Define commands
 \newcommand{\half}{\ensuremath{\frac{1}{2}}}

 \newcommand{\bea}{\begin{eqnarray}}
 \newcommand{\eea}{\end{eqnarray}}
 \newcommand{\beq}{\begin{equation}}
 \newcommand{\eeq}{\end{equation}}
 \newcommand{\bed}{\begin{displaymath}}
 \newcommand{\eed}{\end{displaymath}}

 \newcommand{\pd}[2]{\frac{\partial #1}{\partial #2}}
 \newcommand{\pf}[2]{\frac{d #1}{d #2}}
 \newcommand{\pdt}[2]{\frac{\partial^2 #1}{\partial #2^2}}
 \newcommand{\pft}[2]{\frac{d^2 #1}{d #2^2}}
 \newcommand{\pdtno}[2]{\frac{\partial^2 #1}{\partial #2}}
 \newcommand{\pdd}[3]{\frac{\partial^2 #1}{\partial #2 \partial #3}}
 \newcommand{\pff}[3]{\frac{d^2 #1}{d #2 d #3}}

 \graphicspath{{../figures/}}


%%%%%%%%%%%%%%%%%%%%%%%%%%%%%%%%%%%%%%%%%%%%%%%%%%%%%%%%%%%%%%%%
%%%                     Title Info                           %%%
%%%%%%%%%%%%%%%%%%%%%%%%%%%%%%%%%%%%%%%%%%%%%%%%%%%%%%%%%%%%%%%%

\title[\hspace{-0.2cm} DIRK Adjoint]
{
Discrete Adjoint: Newmark--Beta--Gamma (NBG)
}

\author[Komahan Boopathy]
{
  \Large {Komahan Boopathy}\\
}

\institute
{
  \large Georgia Institute of Technology\\
 School of Aerospace Engineering\\
 Atlanta, GA
}

\date
{
\small \today
}

\begin{document}

\begin{frame}
  \titlepage
\end{frame}

%\begin{frame}
%  \frametitle{Outline}
%  \tableofcontents
%\end{frame}

\begin{frame}[allowframebreaks] \frametitle{Newmark--Beta--Gamma Method Adjoint}

  \tiny{
    
    The second--order governing differential equations are posed in the following
    descriptor form at the k-th time step:
    $$ R_k(\ddot{q}_k, \dot{q}_k, q_k, t_k , x) = 0.$$

    We use Newmark--Beta--Gamma (NBG) method to approximate the states:
    $$ S_k =  \dot{q}_{k-1}  + (1-\gamma) h \ddot{q}_{k-1} +  \gamma h \ddot{q}_{k}- \dot{q}_k=0$$ 
    $$ T_k =  {q}_{k-1}  + h \dot{q}_{k-1} +\frac{1-2\beta}{2} h^2\ddot{q}_{k-1}+ \beta h^2 \ddot{q}_k-{q}_k=0.$$
    The acceleration states $\ddot{q}_k$ are the primary unknown at each time step. We introduce $\lambda_k$, $\psi_k$ and $\phi_k$ as adjoint variables
    associated with each of these equations. The Lagrangian function is written
    as:
    
    $${\cal{L}} = \sum_{k=0}^N h f_k(\underline{\ddot{q}_k},
    \dot{q}_k, q_k, x) + \sum_{k=0}^N h \lambda_k^T
    R_k(\underline{\ddot{q}_k}, \dot{q}_k, q_k, x) + \sum_{k=0}^N \psi_k
    S_k(\underline{\dot{q}_k}, \dot{q}_{k-1}, \ddot{q}_{k-1},
    \ddot{q}_{k}) + \sum_{k=0}^N \phi_k T_k(\underline{{q}_k},
         {q}_{k-1}, \ddot{q}_{k-1}, \ddot{q}_{k}). $$
         
         The underlined variables denote the primary unknown in each
         equation.
         
         \framebreak
         
         Setting $\pd{\cal{L}}{\dot{q}_k} = 0$ yields:
         \begin{equation}
           \begin{split}
             \psi_k = \psi_{k+1} + h \phi_{k+1}  + \left\{ h \pd{f_{k+1}}{\dot{q}_{k+1}} +  h^2 \pd{f_{k+1}}{{q}_{k+1}} \right\}^T + \left[ h \pd{R_{k+1}}{\dot{q}_{k+1}} +  h^2 \pd{R_{k+1}}{{q}_{k+1}} \right]^T \lambda_{k+1} 
           \end{split}
         \end{equation}


         Setting $\pd{\cal{L}}{{q}_k} = 0$ yields:
         \begin{equation}
           \begin{split}
             \phi_k = \phi_{k+1} + \left\{h \pd{f_{k+1}}{{q}_{k+1}} \right\}^T + \left[ h \pd{R_{k+1}}{{q}_{k+1}} \right]^T \lambda_{k+1}  
           \end{split}
         \end{equation}
         
         Setting $\pd{\cal{L}}{\ddot{q}_k} = 0$ yields:
         \begin{equation}
           \begin{split}
             \left[ \pd{R_k}{\ddot{q}_k} + \gamma h \pd{R_k}{\dot{q}_k} + \beta h^2 \pd{R_k}{{q}_k} \right]^T \lambda_k = &- \left\{ \pd{f_k}{\ddot{q}_k} + \gamma h \pd{f_k}{\dot{q}_k} + \beta h^2 \pd{f_k}{{q}_k} \right\}^T \\
             & -  \frac{1}{h}\left\{  \gamma h  \psi_k + \beta h^2   \phi_k \right\}^T\\
             & -  \left\{ (1-\gamma) h \pd{f_{k+1}}{\dot{q}_{k+1}} + \frac{1-2\beta}{2} h^2 \pd{f_{k+1}}{{q}_{k+1}} \right\}^T \\
                          & -  \left[ (1-\gamma) h \pd{R_{k+1}}{\dot{q}_{k+1}} + \frac{1-2\beta}{2} h^2 \pd{R_{k+1}}{{q}_{k+1}} \right]^T\lambda_{k+1} \\
             & -  \frac{1}{h} \left\{ (1-\gamma) h \psi_{k+1} + \frac{1-2\beta}{2} h^2 \phi_{k+1} \right\}^T\\
           \end{split}
         \end{equation}

         We can equivalently scale the equation with $1/h^2$ and represent as follows:
         
         \begin{equation}
           \begin{split}
             \left[ \frac{1}{h^2} \pd{R_k}{\ddot{q}_k} + \gamma \frac{1}{h} \pd{R_k}{\dot{q}_k} + \beta \pd{R_k}{{q}_k} \right]^T \lambda_k = &- \left\{ \frac{1}{h^2}  \pd{f_k}{\ddot{q}_k} + \gamma \frac{1}{h} \pd{f_k}{\dot{q}_k} + \beta \pd{f_k}{{q}_k} \right\}^T \\
             & -  \frac{1}{h}\left\{  \gamma \frac{1}{h}  \psi_k + \beta   \phi_k \right\}^T\\
             & -  \left\{ (1-\gamma) \frac{1}{h} \pd{f_{k+1}}{\dot{q}_{k+1}} + \frac{1-2\beta}{2} \pd{f_{k+1}}{{q}_{k+1}} \right\}^T \\
                          & -  \left[ (1-\gamma) \frac{1}{h} \pd{R_{k+1}}{\dot{q}_{k+1}} + \frac{1-2\beta}{2} \pd{R_{k+1}}{{q}_{k+1}} \right]^T\lambda_{k+1} \\
             & -  \frac{1}{h} \left\{ (1-\gamma) \frac{1}{h} \psi_{k+1} + \frac{1-2\beta}{2} \phi_{k+1} \right\}^T\\
           \end{split}
         \end{equation}

         \framebreak
        
         Grouping the terms together we get:

         \begin{equation}
           \begin{split}
             \left[ \frac{1}{h^2} \pd{R_k}{\ddot{q}_k} + \gamma \frac{1}{h} \pd{R_k}{\dot{q}_k} + \beta \pd{R_k}{{q}_k} \right]^T \lambda_k = &- \left\{ \frac{1}{h^2}  \pd{f_k}{\ddot{q}_k} + \gamma \frac{1}{h} \pd{f_k}{\dot{q}_k} + \beta \pd{f_k}{{q}_k} \right\}^T \\
             & -  \left\{  \gamma \frac{1}{h}  \pd{f_{k+1}}{\dot{q}_{k+1}} +  \gamma \pd{f_{k+1}}{{q}_{k+1}} \right\}^T \\ 
             & -  \left[  \gamma \frac{1}{h}  \pd{R_{k+1}}{\dot{q}_{k+1}} + \gamma \pd{R_{k+1}}{{q}_{k+1}} \right]^T \lambda_{k+1}  \\
             & -  \left\{ \beta \pd{f_{k+1}}{{q}_{k+1}} \right\}^T \\ 
             & - \left[  \beta\pd{R_{k+1}}{{q}_{k+1}} \right]^T \lambda_{k+1}  \\
             & -  \left\{ (1-\gamma) \frac{1}{h} \pd{f_{k+1}}{\dot{q}_{k+1}} + (\frac{1}{2}-\beta) \pd{f_{k+1}}{{q}_{k+1}} \right\}^T \\
             & -  \left[ (1-\gamma) \frac{1}{h} \pd{R_{k+1}}{\dot{q}_{k+1}} + (\frac{1}{2}-\beta) \pd{R_{k+1}}{{q}_{k+1}} \right]^T\lambda_{k+1} \\
             & -  \frac{1}{h} \left\{ \frac{1}{h} \psi_{k+1} + (\frac{1}{2} +\gamma) \phi_{k+1} \right\}^T\\
           \end{split}
         \end{equation}
        
         \begin{equation}
           \begin{split}
             \left[ \frac{1}{h^2} \pd{R_k}{\ddot{q}_k} + \gamma \frac{1}{h} \pd{R_k}{\dot{q}_k} + \beta \pd{R_k}{{q}_k} \right]^T \lambda_k = &- \left\{ \frac{1}{h^2}  \pd{f_k}{\ddot{q}_k} + \gamma \frac{1}{h} \pd{f_k}{\dot{q}_k} + \beta \pd{f_k}{{q}_k} \right\}^T \\
             & -  \left\{  \frac{1}{h} \pd{f_{k+1}}{\dot{q}_{k+1}} +  (\frac{1}{2} +\gamma) \pd{f_{k+1}}{{q}_{k+1}} \right\}^T \\
             & -  \left[  \frac{1}{h} \pd{R_{k+1}}{\dot{q}_{k+1}} +  (\frac{1}{2} +\gamma)  \pd{R_{k+1}}{{q}_{k+1}} \right]^T\lambda_{k+1} \\
             & -  \frac{1}{h} \left\{ \frac{1}{h} \psi_{k+1} + (\frac{1}{2} +\gamma) \phi_{k+1} \right\}^T\\
           \end{split}
         \end{equation}


         The total derivative can be computed as follows:
         $$\pd{\cal{L}}{x} = \pd{F}{x} = \sum_{k=0}^N h \pd{f_k}{x}^T + \sum_{k=0}^N h
         \pd{R_k}{x}^T \lambda_k.$$
  }
  
\end{frame}

\end{document}

%         We explore further simplifications by substituting equations.
%
%         \begin{equation}
%           \begin{split}
%             \left[ \frac{1}{h^2} \pd{R_k}{\ddot{q}_k} + \gamma \frac{1}{h} \pd{R_k}{\dot{q}_k} + \beta \pd{R_k}{{q}_k} \right]^T \lambda_k = &- \left\{ \frac{1}{h^2}  \pd{f_k}{\ddot{q}_k} + \gamma \frac{1}{h} \pd{f_k}{\dot{q}_k} + \beta \pd{f_k}{{q}_k} \right\}^T \\
%             & -  \frac{1}{h}\left\{  \gamma \frac{1}{h}  \left( \psi_{k+1} + h \phi_{k+1}  + \left\{ h \pd{f_{k+1}}{\dot{q}_{k+1}} +  h^2 \pd{f_{k+1}}{{q}_{k+1}} \right\}^T + \left[ h \pd{R_{k+1}}{\dot{q}_{k+1}} +  h^2 \pd{R_{k+1}}{{q}_{k+1}} \right]^T \lambda_{k+1}  \right) \right\}^T \\
%             & -  \frac{1}{h}\left\{  \beta  \left( \phi_{k+1} + \left\{h \pd{f_{k+1}}{{q}_{k+1}} \right\}^T + \left[ h \pd{R_{k+1}}{{q}_{k+1}} \right]^T \lambda_{k+1}  \right) \right\}^T\\
%             & -  \left\{ (1-\gamma) \frac{1}{h} \pd{f_{k+1}}{\dot{q}_{k+1}} + \frac{1-2\beta}{2} \pd{f_{k+1}}{{q}_{k+1}} \right\}^T \\
%             & -  \left[ (1-\gamma) \frac{1}{h} \pd{R_{k+1}}{\dot{q}_{k+1}} + \frac{1-2\beta}{2} \pd{R_{k+1}}{{q}_{k+1}} \right]^T\lambda_{k+1} \\
%             & -  \frac{1}{h} \left\{ (1-\gamma) \frac{1}{h} \psi_{k+1} + \frac{1-2\beta}{2} \phi_{k+1} \right\}^T\\
%           \end{split}
%         \end{equation}
%
%         \framebreak
