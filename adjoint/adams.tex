\documentclass{beamer}

%%%%%%%%%%%%%%%%%%%%%%%%%%%%%%%%%%%%%%%%%%%%%%%%%%%%%%%%%%%%%%%%
%%%                  Themes and such                         %%%
%%%%%%%%%%%%%%%%%%%%%%%%%%%%%%%%%%%%%%%%%%%%%%%%%%%%%%%%%%%%%%%%
\mode<presentation>
{
  %\usetheme{Copenhagen}  
  %\usetheme{Warsaw}  
  \usetheme{Malmoe}  
%    \setbeamertemplate{headline}{}
  %make my huge toc fit on one slide (and not look horrible)
  %\setbeamerfont{subsection in toc}{series=\bfseries}
  %\setbeamerfont{subsection in toc}{size=\tiny,series=\bfseries}
}

%%%%%%%%%%%%%%%%%%%%%%%%%%%%%%%%%%%%%%%%%%%%%%%%%%%%%%%%%%%%%%%%
%%%                       Packages                           %%%
%%%%%%%%%%%%%%%%%%%%%%%%%%%%%%%%%%%%%%%%%%%%%%%%%%%%%%%%%%%%%%%%
\usepackage{multimedia}
\usepackage{multirow}
\usepackage{subfigure}
\usepackage{amsmath}

% Define commands
 \newcommand{\half}{\ensuremath{\frac{1}{2}}}

 \newcommand{\bea}{\begin{eqnarray}}
 \newcommand{\eea}{\end{eqnarray}}
 \newcommand{\beq}{\begin{equation}}
 \newcommand{\eeq}{\end{equation}}
 \newcommand{\bed}{\begin{displaymath}}
 \newcommand{\eed}{\end{displaymath}}

 \newcommand{\pd}[2]{\frac{\partial #1}{\partial #2}}
 \newcommand{\pf}[2]{\frac{d #1}{d #2}}
 \newcommand{\pdt}[2]{\frac{\partial^2 #1}{\partial #2^2}}
 \newcommand{\pft}[2]{\frac{d^2 #1}{d #2^2}}
 \newcommand{\pdtno}[2]{\frac{\partial^2 #1}{\partial #2}}
 \newcommand{\pdd}[3]{\frac{\partial^2 #1}{\partial #2 \partial #3}}
 \newcommand{\pff}[3]{\frac{d^2 #1}{d #2 d #3}}

 \graphicspath{{../figures/}}


%%%%%%%%%%%%%%%%%%%%%%%%%%%%%%%%%%%%%%%%%%%%%%%%%%%%%%%%%%%%%%%%
%%%                     Title Info                           %%%
%%%%%%%%%%%%%%%%%%%%%%%%%%%%%%%%%%%%%%%%%%%%%%%%%%%%%%%%%%%%%%%%

\title[\hspace{-0.2cm} DIRK Adjoint]
{
Discrete Adjoint: Adams--Bashforth--Moulton (ABM)
}

\author[Komahan Boopathy]
{
  \Large {Komahan Boopathy}\\
}

\institute
{
  \large Georgia Institute of Technology\\
 School of Aerospace Engineering\\
 Atlanta, GA
}

\date
{
\small \today
}

\begin{document}

\begin{frame}
  \titlepage
\end{frame}

%\begin{frame}
%  \frametitle{Outline}
%  \tableofcontents
%\end{frame}

\begin{frame}[allowframebreaks] \frametitle{Time Integration}

  \tiny{

    The second--order governing differential equations are posed in the following
    descriptor form at the k-th time step:
    $$ R_k(\underline{\ddot{q}_k}, \dot{q}_k, q_k, t_k , x) = 0.$$

    We use Adams--Bashforth--Moulton (ABM) method to approximate the states:
    $$ S_k =   \dot{q}_{k-1}  + h \sum_{i=0}^{m-1} a_i \ddot{q}_{k-i} - \underline{\dot{q}_k}  = 0 $$ 
    $$ T_k =   {q}_{k-1}  + h \sum_{i=0}^{m-1} a_i \dot{q}_{k-i} - \underline{{q}_k} = 0.$$

    The acceleration states $\ddot{q}_k$ are the primary unknown at each time step. We introduce $\lambda_k$, $\psi_k$ and $\phi_k$ as adjoint variables
    associated with each of these equations. The Lagrangian function is written
    as:
    
    $${\cal{L}} = \sum_{k=0}^N h f_k + \sum_{k=0}^N h \lambda_k^T
    R_k + \sum_{k=0}^N \psi_k S_k + \sum_{k=0}^N \phi_k T_k. $$
         
         The underlined variables denote the primary unknown in each
         equation.

    \framebreak

    The ABM coefficients upto $6^{th}$ order are tabulated.

    \begin{center}
      \begin{tabular}{cccccccc}
        \hline
        m & $i=0$ & 1 & 2 & 3 & 4 & 5 & 6 \\ 
        \hline
        &&&&&&&\\
        1 & 1     &   &   &   &   &   &  \\
        &&&&&&&\\
        2 & $\frac{1}{2}$ & $\frac{1}{2}$ &   &   &   &   &  \\
        &&&&&&&\\
        3 & $\frac{5}{12}$ & $\frac{8}{12}$ & $-\frac{1}{12}$  &   &   &   &  \\
        &&&&&&&\\
        4 & $\frac{9}{24}$ & $\frac{19}{24}$ & $-\frac{5}{24}$  & $\frac{1}{24}$  &   &   &  \\
        &&&&&&&\\
        5 & $\frac{251}{720}$ & $\frac{646}{720}$ & $-\frac{264}{720}$  & $\frac{106}{720}$  & $-\frac{19}{720}$  &   &  \\
        &&&&&&&\\
        6 & $\frac{475}{1440}$ & $\frac{1427}{1440}$ & $-\frac{798}{1440}$  & $\frac{482}{1440}$  & $-\frac{173}{1440}$  &  $\frac{27}{1440}$ &  \\
        &&&&&&&\\
        \hline   
      \end{tabular}
    \end{center}

    \framebreak

    We demonstrate the adjoint devlopment for three steps of time
    integation. \\
    ~\\
    \underline{\textbf{First step}} $(k=1, m=1)$: 
    $$ h \lambda_1^T R_1(\underline{\ddot{q}_1}, \dot{q}_1, q_1, t_1) = 0$$
    $$ h  f_1(\underline{\ddot{q}_1}, \dot{q}_1, q_1, t_1)^T = 0 $$
    $$ \psi_1^T S_1 = \dot{q}_{0}  + h a_0 \ddot{q}_{1} - \dot{q}_1  $$ 
    $$ \phi_1^T T_1 =  {q}_{0}  + h a_0 \dot{q}_{1} - {q}_1$$ 

    \underline{\textbf{Second step}} $(k=2, m=2)$: 
    $$h \lambda_2^T R_2(\underline{\ddot{q}_2}, \dot{q}_2, q_2, t_2) = 0$$
    $$h f_2(\underline{\ddot{q}_2}, \dot{q}_2, q_2, t_2)^T = 0$$
    $$ \psi_2^T S_2 = \dot{q}_{1}  + h a_0 \ddot{q}_{2} + h a_1 \ddot{q}_{1} - \dot{q}_2 $$ 
    $$ \phi_2^T T_2 = {q}_{1}  + h a_0 \dot{q}_{2}  + h a_1 \dot{q}_{1} - {q}_2$$ 

    \underline{\textbf{Third step}} $(k=3, m=3)$: 
    $$h\lambda_3^TR_3(\underline{\ddot{q}_3}, \dot{q}_3, q_3, t_3) = 0$$
    $$hf_3(\underline{\ddot{q}_3}, \dot{q}_3, q_3, t_3)^T = 0$$
    $$ \psi_3^TS_3 = \dot{q}_{2}  + h a_0 \ddot{q}_{3} + h a_1 \ddot{q}_{2}  + h a_2 \ddot{q}_{1} - \dot{q}_3 $$ 
    $$ \phi_3^TT_3 = {q}_{2}  + h a_0 \dot{q}_{3} + h a_1 \dot{q}_{2}  + h a_2 \dot{q}_{1} - {q}_3 $$ 

    \framebreak
   
    Setting $\pd{\cal{L}}{{q}_1} = 0$ yields:
    $$ h \lambda_1^T \pd{R_1}{q_1} + h \pd{f_1}{q_1} + \psi_1^T \pd{S_1}{q_1} + \phi_1^T \pd{T_1}{q_1} = 0 $$

  }

\end{frame}


\end{document}

